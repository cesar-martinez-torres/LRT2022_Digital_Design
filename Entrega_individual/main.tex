\documentclass{article}
\usepackage[english,activeacute]{babel}
\usepackage[latin1]{inputenc}
\usepackage{amsmath,amsfonts,amssymb,amstext,amsthm,amscd}
\usepackage{graphicx}
\usepackage{here} %usar el comandao[H] Para fijar la posición de las imágenes
\usepackage{multicol}
\graphicspath{{Imagenes/}}
 
% Estilo de página y encabezados 
% --------------------------------------------------------------------------------------------------------------------------------------------------------------
\usepackage{fancyhdr}
\setlength{\headheight}{15.2pt}
\usepackage[paperwidth=8.5in, paperheight=11.0in, top=1.0in, bottom=1.0in, left=1.0in, right=1.0in]{geometry}  
\pagestyle{fancyplain}
\fancyhead[R]{LRT2022 Digital design}
\fancyhead[C]{}
\fancyhead[L]{Fall 2022}
\fancyfoot[L]{}
\fancyfoot[C]{\thepage}
\fancyfoot[R]{}
% --------------------------------------------------------------------------------------------------------------------------------------------------------------
% Inicio del documento
\begin{document}
\fancypagestyle{plain}{
   	\renewcommand{\headrulewidth}{1pt}
   	\renewcommand{\footrulewidth}{1pt}
}
\renewcommand{\footrulewidth}{1pt}
\renewcommand{\tablename}{Tabla}
% --------------------------------------------------------------------------------------------------------------------------------------------------------------
%Titulo y autor
\author{}% No llenar, el documento debe ser anónimo..
\title{Individual assignment}
\date{LMT}%Escriba su carrera
\maketitle
% --------------------------------------------------------------------------------------------------------------------------------------------------------------
% Escribir resumen 150-200 palabras
\begin{abstract}
Este es el abstract.              
\end{abstract}

% --------------------------------------------------------------------------------------------------------------------------------------------------------------
\begin{multicols}{2} %Documento a dos columnas
\section*{Section}\label{seccion}  
Esta es una seccion
\subsection*{Subsection}\label{subseccion}                              	% 
Estas es una subseccion
Texto en italicas \textit{italicas}

\begin{figure}[H]
	\centering
	\includegraphics[scale = 0.25]{1.png}
	\caption{Titulo de la imagen}
	\label{fig:label_fig}
\end{figure}

Texto en negritas \textbf{negritas}

\begin{itemize}
	\item item
	\item item
	\item item
	\item item
\end{itemize}

\begin{enumerate}
    \item item 
    \item item
    \item item
    \item item
\end{enumerate}


Comando para citar  \cite{Johana}

% Bibliografía 
%---------------------------------------------------------------------------------------------------------------------------------------------------------------
\begin{thebibliography}{9}		

\bibitem{Johana}
   Johana, H.(2009)
   \emph{Lineas Equipotenciales}.
   Recuperado de: https://es.slideshare.net/guestd93ebf/infome-2-lineas-equipotenciales-y-campo-electrico

 \bibitem{Teresa}
   Teresa, B. (2015)
   \emph{Electo Est\'atica}.
   Recuperado de: http://www2.montes.upm.es/
   
\end{thebibliography}

% ----------------------------------------------------------------------------------------------------------------------------------------------------------------
\end{multicols}
\end{document}									
% Fin del documento
